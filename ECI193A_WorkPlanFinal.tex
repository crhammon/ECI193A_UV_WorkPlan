\documentclass[11pt,letterpaper,final]{report}
\usepackage[utf8]{inputenc}
\usepackage[english]{babel}
\usepackage{amsmath}
\usepackage{indentfirst}
\usepackage{amsfonts}
\addto{\captionsenglish}{\renewcommand{\bibname}{\LARGE{References}}}
\usepackage{pdfpages}
\usepackage{subcaption}
\usepackage{etoolbox}
\usepackage{amssymb}
\usepackage[font=small,labelfont=bf]{caption}
\usepackage{hanging}
\usepackage{notoccite}
\usepackage{float}
\usepackage{makeidx}
\usepackage{color}
\usepackage{titlesec}
\usepackage{gensymb}
\usepackage{hyperref}
\usepackage{graphicx}
\usepackage{lmodern}

\makeatletter
% section from book
%\newcommand\section{\@startsection {section}{1}{\z@}%
%                                   {-3.5ex \@plus -1ex \@minus -.2ex}%
%                                   {2.3ex \@plus.2ex}%
%                                   {\normalfont\Large\bfseries}}
\renewcommand\chapter{\@startsection {chapter}{0}{\z@}%
                                   {-4.5ex \@plus -1ex \@minus -.2ex}%
                                   {3.3ex \@plus.2ex}% 
                                   {\normalfont\LARGE\textbf}}


\makeatletter



\setlength{\parskip}{\baselineskip}%
\setlength{\parindent}{0pt}%






\usepackage{fourier}
\usepackage[left=1in,right=1in,top=1in,bottom=1in]{geometry}

\titleformat{\chapter}{\normalfont\huge}{\thechapter.}{20pt}{\huge\bf}
\titlespacing*{\chapter}{0pt}{-.5in}{20pt}
\author{Charles Hammond \\ Jaime Luo \\ Jacob Newman}


\begin{document}



\begin{center}

\begin{huge} 

\begin{Huge}\textbf{Work Plan} \end{Huge} \\~\\  UV System Replacement/UVT Requirement Reduction \\~\\
\end{huge}


\begin{Large} \textit{pHlux Engineering} \end{Large}



\begin{large}

\vspace{100pt}
\includegraphics[height=1in]{WP}\\
\end{large}

\vspace{100pt}


\begin{tabular}{ll}
    \textbf{Prepared by:}&Charles Hammond\\
                         & Jaime Luo\\
                         & Jacob Newman\\
                         & \\~\\
    \textbf{Prepared for:}& Colleen Bronner\\
                         & \\~\\
    \textbf{Finalized on:}& \today\\ 

\end{tabular}

\end{center}

\pagenumbering{gobble}


\newpage
\setcounter{page}{0}

\chapter*{Executive Summary}
\pagenumbering{roman} 

    \addcontentsline{toc}{chapter}{Executive Summary}

    This report contains the storm water management design requested by the city of Davis, California, on April 18th, 2018. The sources of stormwater are a tarmac, a grass-covered park, and a housing developement. The design includes a conveyance channel, gutters, inlets and a storm sewer for the housing development, a culvert, and a detention pond. The design is based on a 10-year storm.

Determining the peak runoff flowrate from the watershed is perhaps the most important aspect of the stormwater design process, as nearly every other component of the design depends on it. The peak runoff flowrate of 46.1 cfs was determined by taking the average of eleven different approaches using eight different equations. This strategy of estimating the peak runoff minimizes the effects of the variability between the many different methods in the literature.

The conveyance channel carries water from the tarmac and the park for 3,500 ft, after which it must also convey the discharge from the housing development for the final 500 ft. Square, concrete channels are inhospitable to residents and wildlife alike, so an irregular, multifunctional channel shape with natural lining materials and a floodplain was chosen for the design. The meandering, low-flow channel will hold 15\% of the peak runoff flowrate, so during non-peak operation residents can make use of a running/bike path that runs the length of the channel. The channel has a grade of 0.15\%, has an initial width of 38 ft, and occupies 4.5 acres of land. The peak water depth is 2.68 ft and 1 ft of freeboard was included. To prevent harm to residents from deep and/or fast moving water, the peak water depth, the peak flow velocity, and the peak Froude number were kept below 3.5 ft, 4 ft/s, and 0.65, respectively. The high-flow portion of channel will be lined with 100\% native and drought tolerant plants chosen to provide habitat for beneficial insects and wildlife, while the low-flow channel will be lined with rip-rap. This will make the channel low maintenance, even in dought, and will mitigate the local heat island effect. Additionally, the rip-rap will help recharge aquifers via infiltration. Construction costs were reduced by using a parabolic low flow channel shape, which is the most efficient hydraulic cross-section, and by lining the channel with rip-rap and native flora, which will require very little maintenance.

The gutter and inlet system detailed in this report consists of a rolling profile of combination inlets in sag. The rolling profile prevents an extreme height difference between the beginning and end of the gutter, therefore also reducing construction costs by reducing the excavation volume. The needed capacity of each gutter was calculated via Wong's method with a composite watershed. Each gutter is 700 ft long and 32 in wide with a peak spread of 31.82 in. The grade is 0.5\% and the cross slope is 8\%. All dimensions comply with HEC-22 recommendations. The combination inlets, which consist of a vertical opening in the gutter and a horizontal grate, are in sweeping configuration; this versatile design greatly reduces the clogging that can plague other designs. The inlet is 6 in tall, a safe height that will prevent children from harming themselves by falling in, and the grate is bicycle friendly. Costs were also reduced by using standard, pre-fabricated gutters and commercially available grates.

The storm sewer is designed to carry the water from the inlets described above. This design reduces the construction costs by using smaller, less expensive, pipes at the beginning where the needed capacity is smaller than at the end of the sewer. To ensure structural safety, the minimum level of cover (crown to surface distance) is 1.36 ft. The pipe sizes to be used are 20 in, 24 in, and 30 in. Each pipe was hydraulically optimized to have a depth/diameter as close to 0.9 as possible. The construction costs are further reduced by using commercially available PVC pipes from JM Eagle.

A road crosses the conveyance channel 3,500 ft downstream of the tarmac and park. A culvert was designed to accommodate this road. Culverts are low-cost alternatives to bridges, as they are highly effective at both providing a crossing and conveying water. The culvert design consists of two 40 ft long, 30 in circular concrete pipes with 45$\degree$ bevels to increase inlet hydraulic efficiency. The pipes are placed at the bottom of the low flow channel to prevent the accumulation of trash and the reproduction of mosquitos. The flow velocities in the pipes were kept between 2-10 ft/s to prevent excessive scour and to prevent people or animals from being pulled in. If the soil is deemed to be suitable, the excavation material near the culvert should be used to construct the culvert to reduce construction costs. Additionally, a standard culvert pipe size and material were used to lower the cost of construction.

To prevent the downstream ecosystems from being overwhelmed by the post-development peak runoff flowrate, a detention pond will be constructed 4000 ft downstream from the inital watersheds. With a pre-development peak runoff flowrate of 30 cfs, and the post-development value of 61 cfs, the required storage is 197,904 ft$^3$. The maximum water depth in the pond is 3.5 ft for 10-year storm and a freeboard of 1 ft is included. The depth of 3.5 ft is shallow enough to prevent residents from harming themselves should they fall into the pond. A circular pond with a flat bottom and an island in the middle was chosen for the design; the circular shape is pleasing to the eye, and the island provides habitat for beneficial insects and adds aesthetic value. The pond will be lined with the same native flora used throughout this design to ensure a seamless and natural-feeling transition between the channel and the pond. A side slope of 3:1 (H:V) was chosen to allow for easy landscaping with native vegetation. Construction costs were reduced by extending the pond with a 3:1 slope to the same height as the top of the channel; excavating the entire area would dramatically increase construction costs. The orifice is located on the opposite side of the pond as the inlet and is sized appropriately to allow the pre-development peak runoff flowrate of 30 cfs to flow through it when the water depth reaches 3.5 ft. Orifice costs were reduced by using a standard size, corrugated metal pipe.

This stormwater infrastructure plan is not only effective and low-cost, it strives to emulate nature, to blend seamlessly with the local flora, and to provide multiple functions to the locals. The channel may be seen as a place of refuge for wildlife and people alike during the dry months, and will serve as an informal greenbelt through the neighborhoods, increasing property values and livability at the same time. Choosing to make infrastructure more sustainable and people-friendly makes people want to live there, something every city desires. Figure 1 shows a schematic of the complete stormwater management system.


    
    
\newpage
\setlength{\parskip}{0pt}%
\tableofcontents
\newpage
\listoffigures

\setlength{\parskip}{\baselineskip}%
\addcontentsline{toc}{chapter}{List of Figures}


\newpage
\chapter*{Nomenclature}
\addcontentsline{toc}{chapter}{Nomenclature}

\begin{table}[htbp]
\centering
\begin{tabular}{ccc}
\textbf{Symbol} & \textbf{Meaning} & \textbf{Units} \\
\hline
S&Overland Slope & \ ft/ft\\
$S_p$ &Overland Slope & \% \\
i & Rainfall intensity & in/hr\\
$i_{SI}$ & Rainfall intensity &mm/hr\\
$t_c$& Time of Concentration& min\\
$t_d$& Time of Duration& min\\
$L$& Overland Flow Distance& ft\\
$L_m$& Overland Flow Distance& m\\
$\nu$ & Kinematic Viscocity & $m^2/s$ \\
A & Area & acres\\
$n$& Manning's n &\\
$K_u$& Unit Conversion =1 (3.28 for English)& \\
$k$& Intercept Coefficient& \\
$c$& Roughness Coefficient & \\
$K$& Kirpich Constant & \\
$n_k$& Kirpich Constant & \\
C& Rational method coefficient&\\
$C_{w}$& Wong's coefficient& \\
$R_1$& Detention pond bottom radius& \\
$R_b$& Island's bottom radius& \\
$h$ & Detention pond water depth\\
HEC-22 & Hydraulic Engineering Circular 22\\


yd & yard\\
cfs & cubic foot per second\\
V & Velocity & \\ \hline
\end{tabular} 
\end{table}


\newpage
\setcounter{chapter}{0}
\setcounter{figure}{0}
\setcounter{section}{0}



\chapter{Statement of Problem}
\setcounter{page}{0}
\pagenumbering{arabic}
The rainfall and infiltration in the channel are considered negligible except where otherwise specified. The peak flow numbers in this memo are based on a 10-year rainfall event described by Equation 1.1

\begin{equation}
    i=\dfrac{9.742}{\left(t_d^{0.608}+3.533\right)}
\end{equation}

The watersheds considered for this analysis (see Figure 1.1) consist of a 1,980,000 ft$^2$ concrete tarmac and 2,860,000 ft$^2$ of grass covered playing fields. 

\section{Methods}

\begin{figure}[H]
    \centering
    \includegraphics[height=.4\textheight]{RCP.png}

    \caption{Rating curve showing the storage volume as a function of water depth.}
\end{figure}



\section{Conclusion}

Based on the wide variety, yet striking similarity, of results, the average of all the values (except Kirpich, since it is meant for natural watersheds) displayed in Figure 1.2 is a reasonable choice (46.1 cfs). Averaging the values will minimize the effects of the variability between the results. A factor of safety will not be included in this estimate, as the freeboard included in the conveyance channel is intended to address that concern. 



\setcounter{figure}{0}
\setcounter{section}{0} 
\chapter{Project Objectives}
To carry the water from the park and the concrete tarmac to the detention pond located 4,000 ft away, the conveyance channel design summarized in Table 2.1 is proposed. The required peak flow depth, velocity, and Froude number are all intended to create a channel space that is safe for all members of the community. Keeping the Froude number low keeps the energy of the flow low, keeping the velocity and the peak depth low reduces the danger of people being swept away. The freeboard is a safety factor to safeguard against abnormal behavior or more intense storms.
\begin{table}[htbp]
    \centering
    \caption{Recommended design parameters and design constraints.}
    \begin{tabular}{lcc}
    \textbf{Parameter}& \textbf{Design} & \textbf{Constraint}\\
    \hline
    Length (ft) & 4,000 &4,000   \\
    Peak Flow Depth (ft)  & 2.68 &<3.5\\
    Peak Flow Velocity (ft/s) & 1.11 & <4\\
    Peak Froude Number & 0.18 & <0.65\\
    Freeboard (ft) & 1  & 1 \\
    Peak Flow (cfs) & 46.1 & \\
    Lining & Rip-rap and vegetation &  \\
    Slope (ft/ft) & 0.0015 & \\
    Vertical Drop over 3,500 ft (ft)& 5.25 & \\
    Cross Section & Irregular &  \\
    Peak Flow Width (ft) & 35.26 & \\
    Total Initial Channel Width (ft) & 38.8\\ 
    Total Excavation Volume (yd$^3$) & 34,646 \\
    Total Footprint (acres) & 4.59 \\ \hline
    \end{tabular} 
    \end{table}

    Manning's n values for cement, grass, and brick were approximated as unfinished cement, short-grass pasture, and brick with cement mortar. The values were obtained from \cite{White},\cite{USDA}, and\cite{Metcalf}, respectively.

    \begin{figure}[H]
        \centering
        \includegraphics[width=\textwidth]{F1ss2}
        \caption{Schematic of the inlets and the storm sewer pipes. Not to Scale.}
    \end{figure}

\setcounter{figure}{0}
\setcounter{section}{0}
\setcounter{table}{0}
\chapter{Background}


As part of the design, the City of Davis requested a storm sewer gutter and inlet design for a 10 year storm for the urban development next to the channel. The recommended design is a rolling-profile composite gutter with sweeper combined inlets in sag (see Appendix C for technical drawings). This section outlines the requested design features and the resulting final design decisions; all calculations were performed in FlowMaster. Table 3.1 summarizes the design features.

\begin{table}[H]
\centering
\caption{Gutter and inlet specifications and constraints. See Appendix C for details.}
\begin{tabular}{lcc}
\textbf{Feature}&\textbf{Recommended}&\textbf{Constraint}\\ \\
\textbf{Gutter}&&\\
\hline
Discharge&1.86 cfs&1.86 cfs\\
Width& 32 in & Reasonable\\
Spread & 31.82 in& <Width\\
Max Flow Depth& 0.18 ft & \\
Grade & 0.5\%&\\
Cross Slope& 8\%& \\ \\
\textbf{Inlet/Curb}&&\\
\hline
Opening Length&2.0 ft&\\
Opening Height& 0.5 ft&\\
Curb Throat Type&Horizontal&\\
Throat Incline Angle&90 degrees&\\
Local Depression & 3.0 in& \\
Local Depression Width & 3.0 ft &\\
Clogging&10\%&\\ \\
\textbf{Grate}&&\\
\hline
Width& 2.0 ft in& \\
Length & 2.0 ft &\\
Type&P-50 mm (P-1-7/8")&\\ \hline
\end{tabular}
\end{table}
Table 3.2 displays the main cost items for the design. The construction costs are reduced by using the steeper cross slope, which results in a narrower gutter width, and by using the rolling profile, which reduces the necessary excavation volume compared to a gutter built on a single grade. Using standard grate and gutter sizes and types avoids expensive custom-made materials.
\begin{table}[H]
\centering
\caption{Main cost items.}
\begin{tabular}{c}
\textbf{Item}\\
Gutter and Inlet Materials\\
Construction Labor\\ \hline
\end{tabular}
\end{table}



\setcounter{figure}{0}
\setcounter{section}{0}
\setcounter{table}{0}
\chapter{Technical Approach}

To convey the water collected in the gutters and inlets mentioned previously, this section details the design of a storm sewer network. Table 4.1 contains a summary of the design parameters and Appendix D contains the technical drawings.


\begin{table}[H]
\begin{small}
\centering
\caption{Summary of design parameters. $y_n$ is the normal depth at peak flow, and D is the diameter.}
\begin{tabular}{cccccccp{15mm}c}

\textbf{Pipe}&\textbf{Material}&\textbf{Q (cfs)}&\textbf{Velocity (ft/s)}&\textbf{Diameter (in)}&\textbf{Length (ft)} & \textbf{Slope (\%)} &\centering \textbf{Crown Depth (ft)}&$\mathbf{y_n/D}$\\ \hline
P-1& PVC& 3.72 & 2.1& 20 & 1400 & 0.05 & \centering 1.36 & 0.76 \\ 
P-2& PVC &  7.44& 2.6& 24& 1400& 0.06& \centering 1.90& 0.85\\
P-3& PVC& 14.88& 3.25& 30& 100& 0.07& \centering 1.90& 0.88\\ \hline
\end{tabular}
\end{small}
\end{table}


\section{Summary}

This culvert design provides a low-cost alternatives to a bridge, as it is highly effective at both providing a crossing and conveying water. The culvert design consists of two 40 ft long, 30 in circular concrete pipes with 45$\deg$ bevels to increase inlet hydraulic efficiency. The pipes are placed at the bottom of the low flow channel to prevent the accumulation of trash and the reproduction of mosquitos. The flow velocities in the pipes were kept between 2-10 ft/s to prevent excessive scour and to prevent people or animals from being pulled in. If the soil is deemed to be suitable, the excavation material near the culvert should be used to construct the culvert to reduct construction costs. Additionally, a standard culvert pipe size and material were used to lower the cost of construction. \\~\\



\setcounter{figure}{0}
\setcounter{section}{0}
\setcounter{equation}{0}
\setcounter{table}{0}
\chapter{Deliverables}
Runoff flow rates typically increase when a watershed is developed; the grass and soil become covered in tarmac, pavement, and houses, so less water is absorbed into the soil. To prevent downstream natural ecosystems from being overwhelmed by the increased runoff, detention ponds are designed to contain the upstream channel's discharge and to release the same flowrate into the natural ecosystem as was released pre-development. Table 6.2 summarizes the detention pond and orifice design parameters. See Appendix G for additional technical drawings and for the MATLAB code used to calculate volumes.

\begin{table}[H]
    \centering
    \caption{Detention pond design parameters.}
    \begin{tabular}{lcc}
\textbf{Parameter}&\textbf{Value}&\textbf{Constraint}\\ \hline
Maximum Water Depth, 10-year Storm (ft) & 3.5 & 4 \\
Total Pond Depth (ft) & 4.5  & \\
Freeboard (ft) & 1 &  \\
Pond Storage & 200,123.5 & \\ 
Volume of Excavation (ft$^3$) & 707,233.3 & \\
Orifice Area ft$^2$& 2.85 &\\
Orifice Diameter ft & 2 & \\ \hline

\end{tabular} 
\end{table}


\section{Summary}
This detention pond design effectively manages the pre/post-development scenario and prevents downstream ecosystems from being overwhelmed. With a pre-development peak runoff flowrate of 30 cfs and the post-development value of 61 cfs, the required storage is 197,904 ft$^3$. The maximum water depth in the pond is 3.5 ft for 10-year storm and a freeboard of 1 ft is included. The depth of 3.5 ft is shallow enough to prevent residents from harming themselves should they fall into the pond. A circular pond with a flat bottom and an island in the middle was chosen for the design; the circular shape is pleasing to the eye, and the island provides habitat for beneficial insects and adds aesthetic value. The pond will be lined with the same native flora used throughout this design to ensure a seamless and natural-feeling transition between the channel and the pond. Additionally, the natural lining will facilitate infiltration to recharge groundwater supplies. A side slope of 3:1 (H:V) was chosen to allow for easy landscaping with native vegetation. Construction costs were reduced by extending the pond with a 3:1 slope to the same height as the top of the channel; otherwise, excavation costs would be many times larger. The orifice is located on the opposite side of the pond as the inlet and is sized appropriately to allow the pre-development peak runoff flowrate of 30 cfs to flow through it when the water depth reaches 3.5 ft. Orifice costs were reduced by using a standard size, corrugated metal pipe.

\setcounter{figure}{0}
\setcounter{section}{0}
\setcounter{table}{0}
\chapter{Project Management}


\setcounter{figure}{0}
\setcounter{section}{0}
\setcounter{table}{0}
\chapter{Risk Management}

\begin{figure}[H]
    \centering
    \includegraphics[height=.2\textheight]{F3P.png}
    \caption{Impression of what the pond may look like after landscaping. }
\end{figure}


\setcounter{figure}{0}
\setcounter{section}{0}
\setcounter{table}{0}
\chapter{Required Resources}

\setcounter{figure}{0}
\setcounter{section}{0}
\setcounter{table}{0}
\chapter{References}

\setcounter{figure}{0}
\setcounter{section}{0}
\setcounter{table}{0}
\chapter{Appendix A - Resumes}


\renewcommand{\thefigure}{A.\arabic{figure}}
\setcounter{figure}{0}



\newpage
\bibliography{bib} 
\bibliographystyle{ieeetr} 


\end{document}
